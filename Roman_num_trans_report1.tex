\documentclass[10.5pt]{jsarticle}
\usepackage[utf8]{inputenc}
\usepackage{xcolor}
\usepackage{url}
\usepackage{ulem}
\usepackage{enumerate}
\usepackage{fancyhdr}


\begin{document}
\title{平成28年度 ソフトウェア工学 レポート課題 Part1\\「ソフトウェアのマニュアル作成」}
\date{}
\maketitle

\begin{flushright}
{\large 平性28年6月15日}\\
{\large 67140390 平木場 風太}\\
{\large hm14039@student.miyazaki-u.ac.jp}\\
\end{flushright}

\vspace{0.5in}

\begin{large}
\begin{enumerate}
\item 制作スケジュール

理想:6/7(月)〜6/9(木)プログラム作成、6/10(木)〜6/14(火)マニュアル作成

現実:6/12(日)〜6/13(月)プログラム作成、6~14(火)〜6/15(水)マニュアル作成\\

結局スケジュール通りに作業を進めることができなかった。製作開始がひどく遅れてしまった。実際テストも控えており、プログラミング演習の課題もあったが、そもそも5月時点で始めておけば良かった話である。プログラム自体は簡単であったため何とか納期には間に合ったが、マニュアル作成時に、ああしたいこうしたいといったアイデアが次々と浮かび、遅く始めた自分を呪った。実際の仕事だった場合、納期に間に合わないことは許されない上に、余裕があるとソフトウェアの品質を上げることもできるので、余裕を持ったスケジュールを計画・実行していきたい。\\


\item 工夫した箇所

プログラム
\begin{itemize}
\item 入力する値が数字以外だった場合の例外処理を実装した。
\item 今回はソフトウェア制作ということで、普段ならしない、scanf文でのバッファオーバーフロー(バッファオーバーラン)を防ぐための方法を時間をかけて調べて、何度もテストした。結果うまく動いた。
\item ソフトウェア実行時に引数を与えることで、余計なメッセージを出さずに効率よく変換が行えるようにした。これによりシェルスクリプトに組み込むことも用意となった。(ただし、複数の数字をパイプで渡すことができなかった。今後の課題となった。)\\
\end{itemize}
マニュアル
\begin{itemize}
\item 普段ならMSWordを使うが、今回はLaTeXで文書を作った。自分はWordよりLaTeXの方が性に合ってると思った。
\item マニュアルを書くということで、フリーソフトにお馴染みのReadme.txtを複数参考にした。
\item ソースコードを文書内に取り込んだ(listingsというパッケージを使っており、行番号が振られたり、インデントがずれなかったりするだけでなく、ソースファイルに変更があった時、.texファイルをいじらなくても、タイプセットすれば文書内のソースコードが更新されるというメリットもある)から楽だった。
\end{itemize}

\end{enumerate}
\end{large}



\end{document}