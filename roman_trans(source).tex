\documentclass{jsarticle}
\usepackage[utf8]{inputenc}
\usepackage[truedimen,margin=15truemm]{geometry}
\usepackage{listings, jlisting}
\usepackage{xcolor}
\usepackage{url}
\usepackage{ulem}
\usepackage{fancyhdr}
\definecolor{OliveGreen}{cmyk}{0.64,0,0.95,0.40}
\definecolor{colFunc}{rgb}{1,0.07,0.54}
\definecolor{CadetBlue}{cmyk}{0.62,0.57,0.23,0}
\definecolor{Brown}{cmyk}{0,0.81,1,0.60}
\definecolor{colID}{rgb}{0.63,0.44,0}


\lstset{
language={C}, basicstyle={\small},%
identifierstyle={\small},%
commentstyle={\small\ttfamily\color{OliveGreen}},% 
keywordstyle={\small\bfseries},%
ndkeywordstyle={\small},% 
stringstyle={\small\ttfamily\color{CadetBlue}}, frame={tb}, breaklines=true, columns=[l]{fullflexible},% 
numbers=left,% 
xrightmargin=0zw,% 
xleftmargin=3zw,% 
numberstyle={\scriptsize},% 
stepnumber=1, numbersep=1zw,% 
lineskip=-0.5ex% 
}

\pagestyle{fancy}
\lhead{2016/6/29(水)}
\rhead{平木場 風太}
\rfoot{roman\_trans ver2.0}
\cfoot{\thepage}


\begin{document}

\noindent
{\Huge roman\_trans ver2.0}\\ 
{\large【作成者】67140390 平木場 風太}\\
{\large【制作日】2016年6月29日(水曜日)}\\
{\large【連絡先】hm14039@student.miyazaki-u.ac.jp\\
{\large【動作環境】OS X El Capitan(10.11.5)}\\
{\large【開発環境】gcc 4.2.1}\\
{\large【ダウンロード】\url{https://github.com/korosuke613/Roman_num_trans}}\\
{\large【インストール】上のURLでzip形式でファイルをダウンロードし解凍する。Roman\_num\_transという実行ファイルがあるので、カレントディレクトリにて./Roman\_num\_transをコマンドラインで実行することで使用できる。}\\


\section{概要}
このソフトウェアはアラビア数字をローマ数字に変換するソフトウェアです。
\\

\section{使い方}
コマンドラインから実行してください。

パイプまたはリダイレクトによる入力がある場合挙動が異なります。

プログラム名の後に、引数としてオプションを与えると変換の法則が変わります。

\subsection{パイプまたはリダイレクトによる入力を与えない場合}
プログラム実行時に引数を与えない場合、キーボードから入力された数字を変換して出力します。終了コード'q'が入力されるまで繰り返します。\\

動作:
\begin{enumerate}
  \item コマンドライン上でRoman\_num\_transを実行する。
  \item 画面の支持に従い、変換したいアラビア数字をキーボードから入力する。 \label{enum:入力}
  \item 変換されたローマ数字が標準出力される。(異常な値が入力された場合、エラーを出力する。) 
  	\label{enum:出力}
  \item \ref{enum:入力},\ref{enum:出力}を繰り返す。('q'を入力したら終了)\\
\end{enumerate}

(例):入力値 17
\begin{itemize}
\item[] \$ ./Roman\_num\_trans
\item[] ------------------------------------------------------
\item[] アラビア数字を入力してください(qで終了)
\item[] 17
\item[] XVII
\item[] ------------------------------------------------------
\item[] アラビア数字を入力してください(qで終了)
\item[] q
\end{itemize}

\newpage
\subsection{パイプまたはリダイレクトによる入力を与えない場合}
プログラム実行時にパイプまたはリダイレクトによりプログラムに入力を与える場合、与えた数字を1つずつ変換し出力します。全ての数字を処理するまで繰り返します。\\

動作:
\begin{enumerate}
  \item コマンドライン上でRoman\_num\_transを実行する。
  \item 外部から与えたアラビア数字を1つずつローマ数字に変換し、それぞれ標準出力される。(異常な値が入力された場合、エラーメッセージを出力する。) 
  \item 終了\\
\end{enumerate}

(例):入力値 17,350,1298
\begin{itemize}
\item[] \$ echo 17 350 1298 $|$ ./Roman\_num\_trans
\item[] XVII
\item[] CCCL
\item[] MCCXCVIII
\end{itemize}

\subsection{オプション}
-f : ローマ数字の特定の文字を4回異常繰り返しては行けないルールを破棄する。

最小値:1、最大値:49999(50000に当たる文字がないため)

(例):入力値 4,9,4444
\begin{itemize}
\item[] \$ echo 4 9 4444 $|$ ./Roman\_num\_trans -f
\item[] IIII
\item[] VIIII
\item[] MMMMCCCCXXXXIIII\\
\end{itemize}

-g : ローマ数字の特定の文字を4回異常繰り返しては行けないルールを破棄する。\\
~~~~~~~~また、ローマ数字の$5*10^n$に当たる文字を消去する。

最小値:1、最大値:99999(100000に当たる文字がないため)

(例):入力値 5,90,9999
\begin{itemize}
\item[] \$ echo 5 90 9999 $|$ ./Roman\_num\_trans -g
\item[] IIIII
\item[] XXXXXXXXXX
\item[] MMMMMMMMMCCCCCCCCCXXXXXXXXXIIIIIIIII\\
\end{itemize}

-v:プログラムのバージョンを表示する。

\newpage
\section{注意事項}

\begin{itemize}
  \item ローマ数字をアラビア数字に変換することはできません。(これから搭載していきたい)
  \item ローマ数字は一般的な表記法として、1から3999の範囲の数字にしか対応してない\cite{roman_num}のですが、5000と10000に当たる文字\cite{roman_num2}を導入したため、1から39999の範囲となります。
  
  それ以外の数字を入力した場合、エラーを出力します。 
  \item 数字以外の文字を入力した場合、エラーを出力します。
  \item C言語で作られているため、コマンドラインからしか実行できません。  
\end{itemize}

\section{更新履歴}
ver.2.0
\begin{itemize}
  \item パイプ・リダイレクトでの入力に対応しました。そのかわり、引数からの数字の入力ができなくなりました。
  \item ローマ数字の5000,10000に当たる文字\cite{roman_num2}を導入しました。
  \item オプションとして同じ文字を4回以上繰り返せるモードと、$5*10^n$に当たる文字を使わないモードを追加しました。
\end{itemize}

\begin{thebibliography}{9}
\bibitem{roman_num}  ローマ数字は奥が深い・・・3999までは! 

\url{http://plaza.rakuten.co.jp/higashiindo/diary/201207200000/}
\bibitem{roman_num2} ローマ数字 - CyberLibrarian 

\url{http://www.asahi-net.or.jp/~ax2s-kmtn/ref/roman_num.html}
\end{thebibliography}

\newpage
付録[ソースコード]
\lstinputlisting[caption=main.c]{main.c}

\newpage
\lstinputlisting[caption=roman\_common.c]{roman_common.c}

\newpage
\lstinputlisting[caption=Roman\_num\_trans.h]{Roman_num_trans.h}


\end{document}